%
%  untitled
%
%  Created by Sergio on 2009-02-11.
%  Copyright (c) 2009 __MyCompanyName__. All rights reserved.
%
\documentclass[]{article}

% Use utf-8 encoding for foreign characters
\usepackage[utf8]{inputenc}

% Setup for fullpage use
\usepackage{fullpage}

% Uncomment some of the following if you use the features
%
% Running Headers and footers
%\usepackage{fancyhdr}

% Multipart figures
%\usepackage{subfigure}

% More symbols
%\usepackage{amsmath}
%\usepackage{amssymb}
%\usepackage{latexsym}

% Surround parts of graphics with box
\usepackage{boxedminipage}

% Package for including code in the document
\usepackage{listings}

% If you want to generate a toc for each chapter (use with book)
\usepackage{minitoc}

% This is now the recommended way for checking for PDFLaTeX:
\usepackage{ifpdf}

%\newif\ifpdf
%\ifx\pdfoutput\undefined
%\pdffalse % we are not running PDFLaTeX
%\else
%\pdfoutput=1 % we are running PDFLaTeX
%\pdftrue
%\fi

\ifpdf
\usepackage[pdftex]{graphicx}
\else
\usepackage{graphicx}
\fi
\title{Reto 2}
\author{  }

\date{2009-02-11}

\begin{document}

\ifpdf
\DeclareGraphicsExtensions{.pdf, .jpg, .tif}
\else
\DeclareGraphicsExtensions{.eps, .jpg}
\fi

\maketitle

Sergio Botero Uribe. 200710001010.

%\begin{abstract}
%\end{abstract}

\section{Explicaci\'on del ejemplo.}
La simple animaci\'on que hice no la tom\'e de ning\'un ejemplo, simplemente es una pir\'amide con colores y que tiene dos rotaciones sobre dos ejes, nada extra\~no, lo interesante que pude probar fue la forma de hacer los includes para que se complie dependiendo de la plataforma, pero todav\'ia tengo algo para revisar porque los ejes de la rotaci\'on no quedaban declarados como lo hab\'ia hecho al principio.
\bibliographystyle{plain}
\bibliography{}
\end{document}
