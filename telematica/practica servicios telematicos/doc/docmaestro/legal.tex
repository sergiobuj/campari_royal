\documentclass[
	%draft,
	%submission,
	%compressed,
	final,
	%
	%technote,
	%internal,
	%submitted,
	%inpress,
	%reprint,
	%
	%titlepage,
	notitlepage,
	%anonymous,
	narroweqnarray,
	inline,
	twoside,
         %invited,
	]{ieee}

\newcommand{\latexiie}{\LaTeX2{\Large$_\varepsilon$}}

%Code
	\usepackage{listings}
	\usepackage{fancyvrb}
	\usepackage{moreverb}
	\usepackage{listings}
%	\DefineVerbatimEnvironment{code}{Verbatim}{fontsize=\small}
%	\DefineVerbatimEnvironment{example}{Verbatim}{fontsize=\small}
	\usepackage{amssymb}
	\def\myTabs{2}
%edoC
\usepackage{graphicx}
%Español
	\usepackage[utf8]{inputenc}
	\usepackage[spanish]{babel}
%loñapsE

\begin{document}

\title[Load Balancer]{
       Telemática\\ Load Balancer}


\author[]{Sebastián Arcila Valenzuela (\textit{sarcilav@eafit.edu.co})
\and{}y Sergio Botero Uribe (\textit{sbotero2@eafit.edu.co}).
}

\journal{ST0255-031 Telemática}
\titletext{, \today}
\maketitle               


\section{Legal}

Para el load balancer se tomaron unas clases de una explicación llamada ``Linux Socket Programming In C++, Published in Issue 74 of Linux Gazette, January 2002''  \cite{web} que se modificaron en algunos aspectos para ajustarlas a nuestras necesidades.\\
El código es propiedad de Rob Tougher (2002)\cite{mail} y está bajo la licencia de OPL \cite{opl} que permite modificarlo siempre y cuando se identifique el nombre del autor real y se haga claridad en el hecho de que el código no es propio.\\
Usamos este código porque era el más apropiado para trabajar al nivel de abstracción que queríamos con los sockets en C/C++, con los que al
principio tuvimos inconvenientes.\\ El código nos sirvió para manejar la creación de sockets y manejo de excepciones.\\
Las clases que se usaron fueron: ``ClientSocket'', ``ServerSocket'', ``SocketException'' y ``Socket''.

\begin{thebibliography}{1}


\bibitem{web}
\newblock{http://linuxgazette.net/issue74/tougher.html}\\
\bibitem{mail}
\newblock{Rob Tougher\\ rtougher@yahoo.com\\Copyright © 2002.}\\
\bibitem{opl}
\newblock{http://linuxgazette.net/copying.html}\\

\end{thebibliography}


\end{document}
