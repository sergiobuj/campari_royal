\documentclass[
	%draft,
	%submission,
	%compressed,
	final,
	%
	%technote,
	%internal,
	%submitted,
	%inpress,
	%reprint,
	%
	%titlepage,
	notitlepage,
	%anonymous,
	narroweqnarray,
	inline,
	twoside,
         %invited,
	]{ieee}

\newcommand{\latexiie}{\LaTeX2{\Large$_\varepsilon$}}


\usepackage{graphicx}
%Español
	\usepackage[utf8]{inputenc}
	\usepackage[spanish]{babel}
%loñapsE
%\usepackage{multicol}
\begin{document}

\title[direccionamiento]{
       Telemática\\ Direccionamiento}


\author[]{Sebastián Arcila Valenzuela (\textit{sarcilav@eafit.edu.co})
\and{}y Sergio Botero Uribe (\textit{sbotero2@eafit.edu.co}).
}

\journal{ST0255-031 Telemática}
\titletext{, \today}
\maketitle               

\begin{abstract} 
Documentación de la práctica sobre direccionamiento que se trabajó con el Cisco Packet Tracer 5.0. En el documento se encuentran las direcciones de
cada una de las redes, y la estructura en general de la solución planteada para conectar las ciudades de la forma que se necesitaba.\\
\end{abstract}

\section{Introducción}
	Los esquemas de direccionamiento brindan una mejor visión de la estructura de una empresa o una
organización en general ya que indica en cierta forma la configuración en cuanto jerarquía e
importancia. Un buen esquema da la posibilidad de asignar de forma eficiente las direcciones apropiadas
a las redes que se quieren conectar, así como todos los clientes y los servicios que se tienen.

En esta práctica de direccionamiento se va a trabajar la parte práctica en un simulador que provee Cisco
y que funciona y emula todos los equipos que ofrece esta empresa. Esta es una excelente forma de
familiarizarse con los comandos permitidos por estos equipos y con la forma de interconectar diferentes
redes con sus tablas de direccionamiento estáticas y la configuración de las interfaces.

Un punto adicional de la práctica fue el trabajo con las redes virtuales, para el cual se simuló la conexión
de dos redes virtuales en las cuales se aplicaron los conceptos de `inter vlan routing'. Todo esto aprovechando
el protocolo VTP ({\em VLAN Trunk Protocol} \cite{VTP})implementado por Cisco y que se puede poner en funcionamiento con la herramienta Cisco
Packet Tracer\cite{PT}.

\section{Objetivos}
	\begin{itemize}

\item Relacionarse con el uso de herramientas para configuración de redes.
\item Aprender sobre la configuración de dispositivos Cisco y los diferentes
comandos con los que se manejan.
\item La implementación de redes virtuales con la ayuda de las herramientas
de configuración de redes.
\item Implementación de nubes de Frame-Relay y los conceptos detrás de su
utilidad.
\item Poner en práctica los conceptos sobre VLSM y redes virtuales
adquiridos en clase.

\end{itemize}

\section{Diseño de la solución}
	\input{contenido/solucion}

	\subsection{Configuración de direcciones IP}
		El problema nos presenta cinco ciudades que se tienen que conectar con una estructura determinada. En la estructura
se cuenta con una ciudad sede que se conecta con otras dos mediante una nube para `Frame-Relay', y se conecta a otras
dos ciudades directamente. La ciudad principal incorpora dos redes virtuales (VLAN).
Todas las redes tienen que poder entrar en comunicación con cualquier red en toda la instalación.
El esquema de direcciones que se incorporó tiene en cuenta los siguientes aspectos:

\begin{center}
\begin{tabular}{ l r }
Ciudad &  Número de hosts \\ \hline
Medellín  & 250* \\
 \raggedright vlan1 & 126 \\
 vlan2 & 62 \\
  vlan3 & 62 \\
Cali  & 126 \\
Bucaramanga  & 126 \\
Pereira  & 62 \\
Armenia  & 62 \\
\end{tabular}
\end{center}


En base a estas suposiciones sobre el número de hosts para cada red, se partió de la dirección \verb=192.168.0.0 /22= para
obtener todas las subredes. Se utilizó VLSM y el resultado se encuentra en la siguiente tabla de direcciones ({\em ver página
siguiente}).\\ Se usó una dirección IP privada clase C que para los fines y el tamaño de nuestro problema era la más ajustada y además no 
dejaba una gran cantidad de direcciones sin utilizar.


\begin{center}
\begin{table*}[htpd]
\hfill{}
\begin{tabular}{l lclc}
Ciudad & Dirección de red & Máscara &  Broadcast & Rango de direcciones hosts\\ \hline
Medellín & & & \\
{\em vlan1Med} & 	\verb=192.168.0.0= &	\verb=/25= & \verb=192.168.0.127= & 	\verb= 192.168.0.1  -  192.168.0.126= \\
{\em vlan2Med} &	\verb=192.168.0.128= & 	\verb=/26= & \verb=192.168.0.191= & 	\verb= 192.168.0.129   -  192.168.0.190= \\
{\em vlan3Med} &	\verb=192.168.0.192= & 	\verb=/26= & \verb=192.168.0.255= & 	\verb= 192.168.0.193   -  192.168.0.254= \\
Cali & 		\verb=192.168.1.0= & 	\verb=/25= & \verb=192.168.1.127= & 	\verb= 192.168.1.1  -  192.168.0.126= \\
Bucaramanga &\verb=192.168.1.128=&	\verb=/25= & \verb=192.168.1.255= &	\verb= 192.168.1.129  -  192.168.1.254= \\
Pereira &  	\verb=192.168.2.0= & 	\verb=/26= & \verb=192.168.2.63= & 	\verb= 192.168.2.1  -  192.168.2.62= \\
Armenia &	\verb=192.168.2.64= & 	\verb=/26= & \verb=192.168.2.127= & 	\verb= 192.168.2.65  -  192.168.2.126= \\
\end{tabular}
\hfill{}
\caption{Tabla de direcciones para las ciudades}
\end{table*}
\end{center}

\begin{center}
\begin{table*}[htdp]
\hfill{}
\begin{tabular}{l lclc}
Nube FR & 	\verb=192.168.2.128= &	\verb=/29= & \verb=192.168.2.135= & 	\verb= 192.168.2.129  -  192.168.2.134= \\
Med - Cali &	\verb=192.168.2.136= & 	\verb=/30= & \verb=192.168.2.139= & 	\verb= 192.168.2.137  -  192.168.2.138= \\
Med - B/manga &\verb=192.168.2.140= & \verb=/30= & \verb=192.168.2.143= & 	\verb= 192.168.2.141  -  192.168.2.142= \\
Med &		\verb=192.168.2.144=&	\verb=/30= & \verb=192.168.2.147= &	\verb= 192.168.2.145  -  192.168.2.146= \\
\end{tabular}
\hfill{}
\caption{Conexiones entre ciudades}
\end{table*}
\end{center}
	\subsection{VLAN}
		Una red virtual me permite tener una o más redes distintas montadas sobre la misma estructura física.

Con el fin de configurar las dos redes virtuales que se debían implementar en la sede de Medellín se
debía configurar cada `switch' y determinar cada puerto de este para una red virtual determinada.
En el modelo se tienen dos redes virtuales configuradas, la primera vlan10 y la segunda vlan20.

La conexión entre `switchs' se debe hacer mediante puertos denominados `trunk', que permiten el
paso de las redes virtuales. El trunk se conecta a otro trunk para establecer la red.

Para mostrar la configuración de una VLAN se conectaron sólo cuatro clientes formando dos redes, los cuales están en la
segunda y tercera interfaz de red de los `switchs'.

	\subsection{Tablas de enrutamiento}
		\begin{table}[h]
%\caption{enrutamientoMed}
\label{enrutamientoArm}
\begin{center}
\title{Tabla de enrutamiento Armenia} 
\begin{tabular}{lcll}
Dirección destino & Máscara & Siguiente salto &  \\ \hline
\verb=192.168.2.136= & \verb=/30= & \verb=192.168.2.129 = & Med \\
\verb=192.168.2.140= & \verb=/30= & \verb=192.168.2.129 = & Med \\
\verb=192.168.2.144= & \verb=/30= & \verb=192.168.2.129 = & Med \\
\verb=192.168.1.128= & \verb=/25= & \verb=192.168.2.129 = & Med \\
\verb=192.168.2.0= & \verb=/26= & \verb=192.168.2.130 = & Per \\
\verb=192.168.1.0= & \verb=/25= & \verb=192.168.2.129 = & Med \\
\verb=192.168.0.0= & \verb=/25= & \verb=192.168.2.129 = & Med \\
\verb=192.168.0.128= & \verb=/26= & \verb=192.168.2.129 = & Med \\
\verb=192.168.0.192= & \verb=/26= & \verb=192.168.2.129 = & Med \\
\end{tabular}
\end{center}
\end{table}%


\begin{table}[h]
%\caption{enrutamientoMed}
\label{enrutamientoBuc}
\begin{center}
\title{Tabla de enrutamiento Bucaramanga}
\begin{tabular}{lcll}
Dirección destino & Máscara & Siguiente salto &  \\ \hline
\verb=192.168.2.136= & \verb=/30= & \verb=192.168.2.141 = & Med \\
\verb=192.168.2.144= & \verb=/30= & \verb=192.168.2.141 = & Med \\
\verb=192.168.2.128= & \verb=/29= & \verb=192.168.2.141 = & Med \\
\verb=192.168.2.0= & \verb=/26= & \verb=192.168.2.141 = & Med \\
\verb=192.168.2.64= & \verb=/26= & \verb=192.168.2.141 = & Med \\
\verb=192.168.1.0= & \verb=/25= & \verb=192.168.2.141 = & Med \\
\verb=192.168.0.0= & \verb=/25= & \verb=192.168.2.141 = & Med \\
\verb=192.168.0.128= & \verb=/26= & \verb=192.168.2.141 = & Med \\
\verb=192.168.0.192= & \verb=/26= & \verb=192.168.2.141 = & Med \\
\end{tabular}
\end{center}
\end{table}%


\begin{table}[h]
%\caption{enrutamientoMed}
\label{enrutamientoCal}
\begin{center}
\title{Tabla de enrutamiento Cali}
\begin{tabular}{lcll}
Dirección destino & Máscara & Siguiente salto &  \\ \hline
\verb=192.168.2.140= & \verb=/30= & \verb=192.168.2.137 = & Med \\
\verb=192.168.2.144= & \verb=/30= & \verb=192.168.2.137 = & Med \\
\verb=192.168.2.128= & \verb=/29= & \verb=192.168.2.137 = & Med \\
\verb=192.168.1.128= & \verb=/25= & \verb=192.168.2.137 = & Med \\
\verb=192.168.2.0= & \verb=/26= & \verb=192.168.2.137 = & Med \\
\verb=192.168.2.64= & \verb=/26= & \verb=192.168.2.137 = & Med \\
\verb=192.168.0.0= & \verb=/25= & \verb=192.168.2.137 = & Med \\
\verb=192.168.0.128= & \verb=/26= & \verb=192.168.2.137 = & Med \\
\verb=192.168.0.192= & \verb=/26= & \verb=192.168.2.137 = & Med \\
\end{tabular}
\end{center}
\end{table}%



\begin{table}[h]
%\caption{enrutamientoMed}
\label{enrutamientoMed}
\begin{center}
\title{Tabla de enrutamiento Medellín}
\begin{tabular}{lcll}
Dirección destino & Máscara & Siguiente salto &  \\ \hline
\verb=192.168.1.128= & \verb=/25= & \verb=192.168.2.142 = & B/ga \\
\verb=192.168.2.0= & \verb=/26= & \verb=192.168.2.130 = & Per \\
\verb=192.168.2.64= & \verb=/26= & \verb=192.168.2.131 = & Arm \\
\verb=192.168.1.0= & \verb=/25= & \verb=192.168.2.138 = & Cal \\
\verb=192.168.0.0= & \verb=/25= & \verb=192.168.2.146 = & SMed \\
\verb=192.168.0.128= & \verb=/26= & \verb=192.168.2.146 = & SMed \\
\verb=192.168.0.192= & \verb=/26= & \verb=192.168.2.146 = & SMed \\
\end{tabular}
\end{center}
\end{table}%


\begin{table}[h]
%\caption{enrutamientoMed}
\label{enrutamientoSwMed}
\begin{center}
\title{Tabla de enrutamiento Multilayer Switch Medellín}
\begin{tabular}{lcll}
Dirección destino & Máscara & Siguiente salto &  \\ \hline
\verb=192.168.2.136= & \verb=/30= & \verb=192.168.2.145 = & Med \\
\verb=192.168.2.140= & \verb=/30= & \verb=192.168.2.145 = & Med \\
\verb=192.168.2.128= & \verb=/29= & \verb=192.168.2.145 = & Med \\
\verb=192.168.1.128= & \verb=/25= & \verb=192.168.2.145 = & Med \\
\verb=192.168.2.0= & \verb=/26= & \verb=192.168.2.145 = & Med \\
\verb=192.168.2.64= & \verb=/26= & \verb=192.168.2.145 = & Med \\
\verb=192.168.1.0= & \verb=/25= & \verb=192.168.2.145 = & Med \\
\end{tabular}
\end{center}
\end{table}%

\begin{table}[h]
%\caption{enrutamientoMed}
\label{enrutamientoPer}
\begin{center}
\title{Tabla de enrutamiento Pereira}
\begin{tabular}{lcll}
Dirección destino & Máscara & Siguiente salto &  \\ \hline
\verb=192.168.2.136= & \verb=/30= & \verb=192.168.2.129 = & Med \\
\verb=192.168.2.140= & \verb=/30= & \verb=192.168.2.129 = & Med \\
\verb=192.168.2.144= & \verb=/30= & \verb=192.168.2.129 = & Med \\
\verb=192.168.1.128= & \verb=/25= & \verb=192.168.2.129 = & Med \\
\verb=192.168.2.64= & \verb=/26= & \verb=192.168.2.131 = & Arm \\
\verb=192.168.1.0= & \verb=/25= & \verb=192.168.2.129 = & Med \\
\verb=192.168.0.0= & \verb=/25= & \verb=192.168.2.129 = & Med \\
\verb=192.168.0.128= & \verb=/26= & \verb=192.168.2.129 = & Med \\
\verb=192.168.0.192= & \verb=/26= & \verb=192.168.2.129 = & Med \\
\end{tabular}
\end{center}
\end{table}%

%
%\begin{table}[htdp]
%%\caption{enrutamientoMed}
%\label{enrutamientoArm}
%\begin{center}
%\title{Tabla de enrutamiento Armenia} 
%\begin{tabular}{ l c l l }
%Dirección de red destino & Máscara & Siguiente salto &  \\ \hline
%\end{tabular}
%\end{center}
%\end{table}%

%\section{Resultados}
	\input{contenido/resultados}

\section{Conclusiones}
	Después de acostumbrarse a la versión emulada del Cisco Packet Tracer hay que enfatizar
que la herramienta brinda todas facilidades para la implementación y configuración de una
red a cualquier escala y con una gran variedad de dispositivos que gracias a las diferentes
interfaces que tiene la herramienta se puede simular de una forma muy cercana a la real
el trabajo con los dispositivos.\\
El hecho de haber conectado dispositivos siguiendo un esquema preestablecido con los
métodos vistos en clase demuestra al importancia de los conceptos para poder crear
redes sin conflictos y perfectamente funcionales desde la primera configuración.
	
\begin{thebibliography}{1}

\bibitem{VTP} Virtual LAN Trunk Protol, Cisco Systems,
\newblock{http://www.cisco.com/en/US/tech/tk389/tk689/technol\\ogies\_tech\_note09186a0080094c52.shtml}\\
\bibitem{PT} Cisco Packet Tracer, Cisco Systems, Networking Academy.
\newblock{http://www.cisco.com/web/learning/netacad/landing/P\\acket\_Tracer.html}\\
\bibitem{ref3}
\newblock{ }\\

\end{thebibliography}


\end{document}
