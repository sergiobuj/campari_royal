Una red virtual me permite tener una o más redes distintas montadas sobre la misma estructura física.

Con el fin de configurar las dos redes virtuales que se debían implementar en la sede de Medellín se
debía configurar cada `switch' y determinar cada puerto de este para una red virtual determinada.
En el modelo se tienen dos redes virtuales configuradas, la primera vlan10 y la segunda vlan20.

La conexión entre `switchs' se debe hacer mediante puertos denominados `trunk', que permiten el
paso de las redes virtuales. El trunk se conecta a otro trunk para establecer la red.

Para mostrar la configuración de una VLAN se conectaron sólo cuatro clientes formando dos redes, los cuales están en la
segunda y tercera interfaz de red de los `switchs'.
