Los esquemas de direccionamiento brindan una mejor visión de la estructura de una empresa o una
organización en general ya que indica en cierta forma la configuración en cuanto jerarquía e
importancia. Un buen esquema da la posibilidad de asignar de forma eficiente las direcciones apropiadas
a las redes que se quieren conectar, así como todos los clientes y los servicios que se tienen.

En esta práctica de direccionamiento se va a trabajar la parte práctica en un simulador que provee Cisco
y que funciona y emula todos los equipos que ofrece esta empresa. Esta es una excelente forma de
familiarizarse con los comandos permitidos por estos equipos y con la forma de interconectar diferentes
redes con sus tablas de direccionamiento estáticas y la configuración de las interfaces.

Un punto adicional de la práctica fue el trabajo con las redes virtuales, para el cual se simuló la conexión
de dos redes virtuales en las cuales se aplicaron los conceptos de `inter vlan routing'. Todo esto aprovechando
el protocolo VTP ({\em VLAN Trunk Protocol} \cite{VTP})implementado por Cisco y que se puede poner en funcionamiento con la herramienta Cisco
Packet Tracer\cite{PT}.