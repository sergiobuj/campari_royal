%
%  untitled
%
%  Created by Sergio on 2009-03-18.
%  Copyright (c) 2009 __MyCompanyName__. All rights reserved.
%
\documentclass[]{article}
% Use utf-8 encoding for foreign characters
\usepackage[utf8]{inputenc}
\usepackage[spanish]{babel}

% Setup for fullpage use
\usepackage{fullpage}

% Uncomment some of the following if you use the features
%
% Running Headers and footers
%\usepackage{fancyhdr}

% Multipart figures
%\usepackage{subfigure}

% More symbols
%\usepackage{amsmath}
%\usepackage{amssymb}
%\usepackage{latexsym}

% Surround parts of graphics with box
\usepackage{boxedminipage}
\parindent=0in
% Package for including code in the document
\usepackage{listings}

% If you want to generate a toc for each chapter (use with book)
\usepackage{minitoc}

% This is now the recommended way for checking for PDFLaTeX:
\usepackage{ifpdf}

%\newif\ifpdf
%\ifx\pdfoutput\undefined
%\pdffalse % we are not running PDFLaTeX
%\else
%\pdfoutput=1 % we are running PDFLaTeX
%\pdftrue
%\fi




\begin{document}
{\Large
Informe de lectura - `A Fast File System for UNIX' {\tiny M. McKusick, W. Joy, S. Leffler, R. Fabry}\\
Sergio Botero Uribe - 200710001010\\
Sistemas Operativos 2010-1 -  \today  \\ \\ 
}
{
\em
%\begin{abstract}
Los avances más significativos de los sistemas operativos se puede decir que son los referentes a la forma en como administran, acceden y guardan las datos en el
disco. Son estas mejoras las que permiten que se tengan sistemas que son eficientes cuando se quiere hacer una transferencia de datos, cuando se quiere replicar
algún tipo de información y cuando se quiere guardar información.
Para los sistemas operativos, mientras más crece el tamaño de los datos, más compleja se vuelve la tarea de tener que buscar espacio en el disco y más veces se debe
 pasar por las vías más lentas de la máquina para obtener esa información, con una implementación pobre o no muy apropiada del sistema de archivos esta tarea
de manipular los datos se vuelve notable para el usuario y a la final puede llegar a limitar el trabajo que se pueda realizar con una determinada implementación de un
sistema de archivos.
Si bien la eficiencia o eficacia de un sistema de archivos la determina el uso que se le dará a la máquina, existen una serie de mejoras que se pueden hacer a nivel de
diseño del sistema de archivos que benefician el rendimiento en general.
El artículo `A Fast File System for UNIX' de M. McKusick, W. Joy, S. Leffler y R. Fabry de 1984 trata sobre las principales mejoras que se hicieron sobre el sistema de archivos
tradicional (el desarrollado en los laboratorios Bell) y explica un poco sobre las razones que motivaron ciertos avances.\\ \\
%\end{abstract}
}
Un disco duro por lo general bajo cualquier sistema de archivos se divide en particiones, donde cada partición tiene un sistema de archivos y cada uno tiene un
súper bloque que almacena información como el número máximo de archivos, el número de bloques y un apuntador a la lista enlazada de espacio libre. Un
sistema de archivos no puede nunca expandirse en dos particiones diferentes. Los datos se almacenan en porciones llamadas bloques, y cada dato dependiendo
del tamaño ocupa uno o varios bloques.\\
Cada archivo que se tiene en disco está referenciado por un `inodo’, los cuales están ubicados al inicio del disco y contienen información como: fecha de modificación,
permisos y propiedad, fecha de creación y la dirección en disco de los bloques que hacen parte del archivo.\\
En este sistema de archivos tradicional se contaba con bloques de 512-bytes, este era también
el tamaño máximo de datos que podía ser transferido a memoria. Otra característica de este sistema es que los bloques libres y los usados nunca se reorganizaban,
por lo que el sistema funcionaba muy bien en el primer periodo de uso pero disminuía su capacidad y velocidad mientras se creaban y eliminaban archivos, esto debido
a que los datos nuevos se localizan
llenando espacios de datos que ya no están, lo que obliga a hacer muchos saltos y búsquedas sobre el disco. Una forma de evitar esta localización aleatoria era ejecutar
un proceso periódicamente que ordenara los bloques ocupados, esto debía ser rediseñado.
\\ \\
El sistema de archivos tradicional evolucionó primero duplicando el tamaño del bloque, pasando de 512 a 1024-bytes, lo que permitió principalmente la transferencia de
archivos completos en un menor número de accesos a disco.
Al tener un tamaño de bloque mayor, se debía hacer algo con los archivos muy pequeños, por lo que se permitió la fragmentación de estos bloques.
Con el hecho de tener bloques más grandes y además fragmentos de estos bloques, surgió la necesidad de optimizar el almacenamiento, donde se resaltan dos cosas: 
se estableció un tamaño estándar para hacer transferencias y escrituras. Y se cambió la forma en que se localizan los datos por una que tiene en cuenta los bloques ya usados.
\\\\
Uno de los grandes saltos que se dio desde el sistema de archivos tradicional fue el de parametrizar el disco y procesador, esto quiere decir que el sistema de archivos
se podía adaptar a las especificaciones de la máquina en la que se ejecutaría, y así podía en base a las capacidades del hardware determinar intervalos de tiempo para
localizar bloques, tamaño de bloques y espacio entre bloques contiguos de datos dependiendo de la velocidad del procesador y de la velocidad de giro de los brazos
del disco. Esto es importante porque la velocidad a la que el procesador recibe datos es diferente a la velocidad a la que el disco puede entregarlos, para hacer esta
operación lo más eficiente posible, los bloques pertenecientes a un archivo se dispondrían de forma que el brazo del disco no tuviera que esperar todo un giro para leer
la información siguiente si no que en el mismo giro pudiera recuperar la mayor cantidad posible simplemente cambiando de pista.\\
El bloque preciso para la siguiente localización también se puede determinar con intervalo de tiempo que puede ser modificado sin necesidad de restaurar el sistema
de archivos, todo para
que un eventual cambio de procesador no requiera un cambio drástico y el sistema de archivos se
va adaptando, este tiempo es simplemente un intervalo de tiempo que el brazo del disco debe esperar para localizar el siguiente bloque y así asegurar que podrá entregar
el contenido a una velocidad a la que pueda ser procesada.
\\\\
Las políticas de localización de datos no son la única forma de potencializar el rendimiento de  
un sistema de archivos, existen también pequeñas modificaciones que permiten que accesos a
disco sean mucho más provechosos. A diferencia del sistema de archivos tradicional, se busca
que los `inodo’ de los archivos contenidos en un directorio se puedan almacenar en un mismo
grupo de cilindros, para que se pueda obtener la información de un directorio sin tener que recorrer el disco dando saltos.
\\\\
Como mejoras funcionales al sistema de archivos se implementaron los `symbolic links’ que simplemente creaban, mediante un ruta relativa o directa, una nueva referencia
a un `inodo’;
al momento de renombrar un archivo o directorio se realiza la verificación de la jerarquía y si
esta no es alterada se copia un temporal y luego se cambia el nombre; los `Long File Names’ que ahora tenían una longitud máxima de caracteres que se podían almacenar
en las propiedades del `inodo’ y también la posibilidad de bloquear un archivo para su acceso, permitiendo que un sólo proceso tenga la posibilidad de modificarlo y usarlo
sin que su información sea modificada por otro proceso, esto se logro mediante los bloqueos fuertes y los bloqueos suaves, que dependiendo de los privilegios del usuario
que los impusiera tenía diferentes ventajas.
\\\\
Estas fueron solo algunas de las modificaciones y mejoras que se le han hecho al sistema de archivos de UNIX a través del tiempo y esperemos que existan muchas por llegar.
Queda muy claro que si bien el sistema operativo tiene mucho trabajo en la manipulación de los
datos, es el sistema de archivos en el que se monte el que permite que muchas de las funciones
tengan un mejor rendimiento frente al mismo sistema operativo que se tenga sobre otro sistema de archivos y mejor aún cuando existe la posibilidad de la parametrizar el
hardware para poder sacarle el mejor provecho a este.

\end{document}
