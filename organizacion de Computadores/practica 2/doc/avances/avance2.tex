%======================================================================
%----------------------------------------------------------------------
%sergiobuj's ieee format template
%======================================================================
\documentclass[%
	%draft,
	%submission,
	%compressed,
	final,
	%
	%technote,
	%internal,
	%submitted,
	%inpress,
	%reprint,
	%
	%titlepage,
	notitlepage,
	%anonymous,
	narroweqnarray,
	inline,
	twoside,
         %invited,
	]{ieee}

\newcommand{\latexiie}{\LaTeX2{\Large$_\varepsilon$}}

%Español
	\usepackage[utf8]{inputenc}
	\usepackage[spanish]{babel}
%loñapsE

%\usepackage{ieeetsp}	% if you want the "trans. sig. pro." style
%\usepackage{ieeetc}	% if you want the "trans. comp." style
%\usepackage{ieeeimtc}	% if you want the IMTC conference style

% Use the `endfloat' package to move figures and tables to the end
% of the paper. Useful for `submission' mode.
%\usepackage {endfloat}

% Use the `times' package to use Helvetica and Times-Roman fonts
% instead of the standard Computer Modern fonts. Useful for the 
% IEEE Computer Society transactions.
%\usepackage{times}
% (Note: If you have the commercial package `mathtime,' (from 
% y&y (http://www.yandy.com), it is much better, but the `times' 
% package works too). So, if you have it...
%\usepackage {mathtime}

% for any plug-in code... insert it here. For example, the CDC style...
%\usepackage{ieeecdc}

\begin{document}

%----------------------------------------------------------------------
% Title Information, Abstract and Keywords
%----------------------------------------------------------------------
\title[Avance  práctica 2]{%
       Avance 2 práctica 2 \\  Organización de Computadores}

% format author this way for journal articles.
% MAKE SURE THERE ARE NO SPACES BEFORE A \member OR \authorinfo
% COMMAND (this also means `don't break the line before these
% commands).
\author[]{Sebastián Arcila Valenzuela (\textit{sarcilav@eafit.edu.co})
\and{}y Sergio Botero Uribe (\textit{sbotero2@eafit.edu.co}).
}

% format author this way for conference proceedings
%\author[PLETT AND KOLL\'{A}R]{%
        %Gregory L. Plett\member{Student Member},\authorinfo{%
        %Department of Electrical Engineering,\\ 
        %Stanford University, Stanford, CA 94305-9510.\\
        %Phone: $+$1\,650\,723-4769, email: glp@simoon.stanford.edu}%
%\and{}and%
%\and{}Istv\'{a}n Koll\'{a}r\member{Fellow}\authorinfo{%
        %Department of Measurement and Instrument Engineering,\\ 
        %Technical University of Budapest, 1521 Budapest, Hungary.\\
        %Phone: $+$\,36\,1\,463-1774, fax: +\,36\,1\,463-4112, 
        %email: kollar@mmt.bme.hu}
%}

\journal{ST0254-031 Organización de Computadores}
\titletext{, \today} 
%\ieeecopyright{0018--9456/97\$10.00 \copyright\ 1997 IEEE}
%\lognumber{xxxxxxx}
%\pubitemident{S 0018--9456(97)09426--6}
%\loginfo{Manuscript received September 27, 1997.}
%\firstpage{0}

%\confplacedate{Ottawa, Canada, May 19--21, 1997}

\maketitle               

\begin{abstract} 
Avance número 2 de la práctica 2, sobre paralelismo.
\end{abstract}

\begin{keywords}
Práctica 2, organización de computadores, paralelismo, Logisim y procesador, ALU, CUDA, sumas parciales.
\end{keywords}

%----------------------------------------------------------------------
% SECTION I: Introduction
%----------------------------------------------------------------------
\section{Comentarios}
\begin{itemize}
\PARstart para esta entrega dos, hemos estado contemplando las diferentes
posibilidades para intentar hacer el código de instrucción, sumar
vector, en un orden inferior $O(n)$, aunque por obvias razones tenemos
que visitar cada elemento del vector al menos una vez tenemos que
empezar a analizar que pasa cuando visitamos varios de ellos en
simultaneo.

\item segmentación:
Podríamos usar segmentación e ir haciendo sumas parciales, donde por
sumas parciales nos referimos en sumar a la vez varias parejas de
elementos, pero siempre tendríamos  el problema que es que a la
siguiente iteración necesitamos valores que todavía se podrían
encontrar en la línea de producción, entonces sería un elemento
obligatorio a revisar el como esto afecta la complejidad del
algoritmo, porque si fuera simplemente ir haciendo bisección y sumando
tendríamos una complejidad de $O(n/16) = O(n)$.

\item Varias ALU.
Es reutilizar la misma idea que usan las tarjetas graficadoras
modernas, y es el uso de varias ALU a la vez, y está idea si que le ha
dado fruto a las graficadoras, pues en computación se utiliza mucho
las operaciones sobre vectores numéricos, como es el caso de los
Shaders (Lenguaje Cg para NVIdia) y de otras tecnologías como CUDA
también de NVidia, en donde lo importante es garantizar la
independencia entre las operaciones atómicas.

\item Varias unidades de control con varias ALU.
La idea sería similar, solo que el en realidad si se podrían hacer las
sumas en el mismo instante y teóricamente cada suma que empiece en
simultaneo con otra debería terminar al mismo tiempo que las demás,
asumiendo que no van a existir fallos de cache. Por lo que el orden
sería $O(log_k n)$ donde $k =$ numero de unidades de control por numero de
ALU por cada unidad de control.
\end{itemize}
%----------------------------------------------------------------------
% SECTION II: Estructura de la práctica
%----------------------------------------------------------------------


%----------------------------------------------------------------------
% SECTION III: Dificultades
%----------------------------------------------------------------------


%----------------------------------------------------------------------
\section{Tareas por realizar}
Lista de las actividades que faltan para terminar la práctica y
que además serían interesantes realizar:
\begin{itemize}
\item Decidir el juego señales
\item Montar el juego de señales
\item Procesador.
\end{itemize}

%----------------------------------------------------


%\begin{thebibliography}{1}

%\end{thebibliography}

%----------------------------------------------------------------------

\end{document}
