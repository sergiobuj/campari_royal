Toda la forma de pensar que se requiere para modelar el procesador y para sincronizar todas las señales en una unidad
de control microprogramada considero que es parte importante en la lógica mínima y general que debe tener un ingeniero
de sistemas.\\ Aunque caímos en el error de dejar la parte de diseño para lo último, se hubiera podido sacar una de tantas
aproximaciones, me parecería muy interesante conocer todas esas formas, para saber cual de todas las ideas de
implementación podían resultar y cuales ofrecen mejor rendimiento.\\
Por último, estoy cansado de usar el mouse y de hacer tanto click.

\begin{flushright} 
	\itshape{-Sergio Botero Uribe}
\end{flushright}