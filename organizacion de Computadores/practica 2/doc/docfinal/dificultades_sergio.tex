El concepto de paralelismo para la suma de los elementos de un vector tiene ciertas implicaciones a nivel
de implementación que desde el principio no se tuvieron claras. También hay que tener en cuenta que
el diseño de un procesador no es como normalmente tengo la costumbre de empezar las prácticas (tipo Extreme
Programming) sino que se debe diseñar primero una aproximación y empezar a crear unidades por partes
separadas para luego unirlas todas. Ese fue el error que tuvo en general el equipo de trabajo para poder
cumplir con todos los puntos de la práctica.

\begin{flushright} 
	\itshape{-Sergio Botero Uribe}
\end{flushright}