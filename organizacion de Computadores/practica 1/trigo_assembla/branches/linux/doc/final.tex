%======================================================================
%----------------------------------------------------------------------
%sergiobuj's ieee format template
%======================================================================
\documentclass[%
	%draft,
	%submission,
	%compressed,
	final,
	%
	%technote,
	%internal,
	%submitted,
	%inpress,
	%reprint,
	%
	%titlepage,
	notitlepage,
	%anonymous,
	narroweqnarray,
	inline,
	twoside,
         %invited,
	]{ieee}

\newcommand{\latexiie}{\LaTeX2{\Large$_\varepsilon$}}
\usepackage{url}
\usepackage{listings}
%Español
	\usepackage[utf8]{inputenc}
	\usepackage[spanish]{babel}
%loñapsE

%\usepackage{ieeetsp}	% if you want the "trans. sig. pro." style
%\usepackage{ieeetc}	% if you want the "trans. comp." style
%\usepackage{ieeeimtc}	% if you want the IMTC conference style

% Use the `endfloat' package to move figures and tables to the end
% of the paper. Useful for `submission' mode.
%\usepackage {endfloat}

% Use the `times' package to use Helvetica and Times-Roman fonts
% instead of the standard Computer Modern fonts. Useful for the 
% IEEE Computer Society transactions.
%\usepackage{times}
% (Note: If you have the commercial package `mathtime,' (from 
% y&y (http://www.yandy.com), it is much better, but the `times' 
% package works too). So, if you have it...
%\usepackage {mathtime}

% for any plug-in code... insert it here. For example, the CDC style...
%\usepackage{ieeecdc}

\begin{document}

%----------------------------------------------------------------------
% Title Information, Abstract and Keywords
%----------------------------------------------------------------------
\title[Entrega final práctica 1]{%
       Entrega final práctica 1 \\  Organización de Computadores}

% format author this way for journal articles.
% MAKE SURE THERE ARE NO SPACES BEFORE A \member OR \authorinfo
% COMMAND (this also means `don't break the line before these
% commands).
\author[]{Sebastián Arcila Valenzuela (\textit{sarcilav@eafit.edu.co}),
\and{} Ruben Dario Bueno Angel (\textit{rbuenoan@eafit.edu.co})
\and{}y Sergio Botero Uribe (\textit{sbotero2@eafit.edu.co}).
}

% format author this way for conference proceedings
%\author[PLETT AND KOLL\'{A}R]{%
        %Gregory L. Plett\member{Student Member},\authorinfo{%
        %Department of Electrical Engineering,\\ 
        %Stanford University, Stanford, CA 94305-9510.\\
        %Phone: $+$1\,650\,723-4769, email: glp@simoon.stanford.edu}%
%\and{}and%
%\and{}Istv\'{a}n Koll\'{a}r\member{Fellow}\authorinfo{%
        %Department of Measurement and Instrument Engineering,\\ 
        %Technical University of Budapest, 1521 Budapest, Hungary.\\
        %Phone: $+$\,36\,1\,463-1774, fax: +\,36\,1\,463-4112, 
        %email: kollar@mmt.bme.hu}
%}

\journal{ST0254-031 Organización de Computadores}
\titletext{, \today}
%\ieeecopyright{0018--9456/97\$10.00 \copyright\ 1997 IEEE}
%\lognumber{xxxxxxx}
%\pubitemident{S 0018--9456(97)09426--6}
%\loginfo{Manuscript received September 27, 1997.}
%\firstpage{0}

%\confplacedate{Ottawa, Canada, May 19--21, 1997}

\maketitle               

\begin{abstract} 
Entrega final de la práctica 1 de Organización de computadores, el informe contiene los detalles de la construcción del módulo así como las especificaciones del módulo, las dificultades encontradas y los comentarios por cada uno de los integrantes del equipo sobre la práctica.
\end{abstract}

\begin{keywords}
Práctica 1, organización de computadores, ensamblador, funciones trigonométricas, intel, nasm
\end{keywords}

\section{Introducción}
``In C++ and Java I experience a certain amount of angst when you ask how to do this and they say, "Well, you do it like this or you could do it like that." There are obviously too many features if you can do something that many ways-and they are more or less equivalent. I think there are smaller concepts that fit better in Inferno.''\cite{cita-thompson}

%----------------------------------------------------------------------
% SECTION I: Detalles de diseño
%----------------------------------------------------------------------
\section{Detalles de diseño de construcción}

Podemos decir que un algoritmo trivial para atacar las series de taylor seria el siguiente, tomemos como ejemplo de aquí en adelante la función seno:
\begin{lstlisting}[language=C++]
float sen(int x, int n)
//n numero de iteraciones 
{
 float xx = to_radianes(x);
 float ans = 0;
 for(int i = 0; i<n;++i)
 {
  ans += pow(-1,i)*pow(xx,2i+1)/(2i+1)!;
 }
 return ans;
}
\end{lstlisting}

Pero como podemos notar hay varias acotaciones que hacer a este código, primero no es necesario tener una función pow, ni una función factorial, puesto que cuando salgo de la iteración $m,(m>0)$ y paso a la iteración $m+1$, tenemos ya calculado hasta $xx^{2m+1}$ y el valor que necesitamos es $xx^{2m+3}$, entonces bastaría con solo multiplicar por $xx^{2}$  a $xx^{2m+1}$ para obtener el valor en $m+1$, casi de igual manera para el factorial cuando entramos en la iteración $m+1$ ya tenemos previamente el valor de $(2m+1)!$, y si nos fijamos el valor que necesitamos es $(2m+3)!$ que es igual a $(2m+1)!(2m+2)(2m+3)$, y para la situación del signo simplemente es en cada iteración hacer $signo = \neg signo$, y voila! no necesitamos funciones externas que desperdicien cálculos y tiempo; obteniendo algo similar a esto:

\begin{lstlisting}[language=C++]
float seno(int x, int n)
{
 float xx=to_radianes(x);
 float ans=0;
 float factorial=1;
 float acum_x2n=xx;
 int sign = -1;
 for(int i = 1; i<n;++i)
 {
   sign *= -1;
   ans += sign*acum_x2n/factorial;
   acum_x2n*=xx*xx;
   factorial*=(2*i)*(2*i+1);
 }
 return ans;
}

\end{lstlisting}
También vale la pena resaltar que los tipos de datos de factorial y de acum\_x2n es de tipo flotante, por que por obvias razones en números enteros da overflow para factorial y ademas los radianes ($xx$) que vamos a usar tienen precisión flotante.

Si lo notamos bien de la misma manera podemos deducir el siguiente algoritmo para coseno:
\begin{lstlisting}[language=C++]
float coseno(int x,int n)
{
 float xx=to_radianes(x);
 float ans=0;
 float factorial=1;
 float acum_x2n=1;
 int sign = -1;
 for(int i = 1; i<n;++i)
 {
  sign *= -1;
  ans += sign*acum_x2n/factorial;
  acum_x2n*=xx*xx;
  factorial*=(2*i-1)*(2*i);
 }
 return ans;
}

\end{lstlisting}

Ahora bien la operación de pasar de grados a radianes es lo más simple de este mundo, es simplemente multiplicar los grados por el factor de conversión que es ${{\pi}\over{180}} = 0.017453293$. Y estás serian todas la anotaciones respecto al diseño del algoritmo.






%----------------------------------------------------------------------
% SECTION II: Estructura de la práctica
%----------------------------------------------------------------------
\section{Especificaciones del procesador}
El procesador implementado tiene dos unidades de control, una que controla el procesador para operaciones normales y
otra unidad de control que se encarga especificamente de la ALU sobrecargada, el cambio de control de las unidades
se hace mediante las señales que salen de la instrucción de suma del vector.\\
Una vez que la unidad de control del procesador pasa el control a su homólogo de la ALU, esta se encarga de traer los
valores del vector de forma consecutiva para guardarlos en los registros de la ALU y cada vez que se traen 4 posiciones
del vector se hace la suma de todos los valores, esto se hace en dos peticiones a la RAM.\\


%----------------------------------------------------------------------
% SECTION III: Dificultades
%----------------------------------------------------------------------
\section{Dificultades encontradas durante el diseño}

%
%  Created by Sergio on 2009-03-18.
%  Copyright (c) 2009 __MyCompanyName__. All rights reserved.
%
%\documentclass[]{article}
%\usepackage[utf8]{inputenc}
%\usepackage[spanish]{babel}
%\usepackage{fullpage}
%\begin{document}

La IDE FASM muy poco amigable y la documentación bastante incompleta, lo deja a uno a la deriva con frecuencia, debido a esto decidimos pasar a un IDE mejor documentado, el NASM, que está bastante maduro y tiene buena documentación en línea.
Documentación general de assembler para tópicos específicos dificil de "googliar", tal vez falta mas familiaridad con la escena ASM.
Una vez se pasó a NASM fué imposible crear la librería DLL para windows con el Linker.exe de Visual Basic, igual la funcionalidad en Windows era un extra pues el pensado era hacerla en Linux en donde quedó funcionando sin problemas.


\begin{flushright} 
	\itshape{-Ruben Dario Bueno Angel}
\end{flushright}
%\end{document}
El concepto de paralelismo para la suma de los elementos de un vector tiene ciertas implicaciones a nivel
de implementación que desde el principio no se tuvieron claras. También hay que tener en cuenta que
el diseño de un procesador no es como normalmente tengo la costumbre de empezar las prácticas (tipo Extreme
Programming) sino que se debe diseñar primero una aproximación y empezar a crear unidades por partes
separadas para luego unirlas todas. Ese fue el error que tuvo en general el equipo de trabajo para poder
cumplir con todos los puntos de la práctica.

\begin{flushright} 
	\itshape{-Sergio Botero Uribe}
\end{flushright}
Logisim es una herramienta aunque de código abierto que ha tenido un
desarrollo muy centralizado y que no es muy flexible y además muy poco
usable, por lo que no nos rindió mucho el trabajo.

\begin{flushright} 
	\itshape{-Sebastián Arcila Valenzuela}
\end{flushright} 
%----------------------------------------------------------------------
\section{Opiniones sobre la práctica}

%
%  Created by Sergio on 2009-03-18.
%  Copyright (c) 2009 __MyCompanyName__. All rights reserved.
%
%\documentclass[]{article}
%\usepackage[utf8]{inputenc}
%\usepackage[spanish]{babel}
%\usepackage{fullpage}
%\begin{document}

%El concepto de paralelismo para la suma de los elementos de un vector tiene ciertas implicaciones a nivel
de implementación que desde el principio no se tuvieron claras. También hay que tener en cuenta que
el diseño de un procesador no es como normalmente tengo la costumbre de empezar las prácticas (tipo Extreme
Programming) sino que se debe diseñar primero una aproximación y empezar a crear unidades por partes
separadas para luego unirlas todas. Ese fue el error que tuvo en general el equipo de trabajo para poder
cumplir con todos los puntos de la práctica.

\begin{flushright} 
	\itshape{-Sergio Botero Uribe}
\end{flushright}


La práctica estuvo divertida e interesante aunque a ratos fué un poco frustrante, trabajar tan a bajo nivel presenta retos que normalmente se ignoran pero una vez superados dejan una sensación de satisfacción en vez de una sensación de rabia, lo cual sucede bastante seguido con lenguajes de alto nivel.

\begin{flushright} 
	\itshape{-Ruben Dario Bueno Angel}
\end{flushright}

%\end{document}

Toda la forma de pensar que se requiere para modelar el procesador y para sincronizar todas las señales en una unidad
de control microprogramada considero que es parte importante en la lógica mínima y general que debe tener un ingeniero
de sistemas.\\ Aunque caímos en el error de dejar la parte de diseño para lo último, se hubiera podido sacar una de tantas
aproximaciones, me parecería muy interesante conocer todas esas formas, para saber cual de todas las ideas de
implementación podían resultar y cuales ofrecen mejor rendimiento.\\
Por último, estoy cansado de usar el mouse y de hacer tanto click.

\begin{flushright} 
	\itshape{-Sergio Botero Uribe}
\end{flushright}
Las herramientas en Java una vez mas me volvieron a sorprender con su
increíble usabilidad y velocidad, es más después de esta práctica
quedamos listo para dictar un curso de paciencia hacia Java. Espero
que esta sea mi ultima experiencia con Java, que invento quien dijo
que Java era un lenguaje de propósito general.
Por otro lado nunca me han gustado los trabajos que sean de desarrollo
en aplicaciones que sean puro click y más clicks.

\begin{flushright} 
	\itshape{-Sebastián Arcila Valenzuela}
\end{flushright}

%----------------------------------------------------


\begin{thebibliography}{2}

\bibitem{Assembla -Free subversion Hosting}
\newblock \url{http://www.assembla.com/spaces/dashboard/index/trigo_assembla},
\newblock Sitio dondese encuentra todo el desarrollo y actividades de la práctica.
\newblock{Assembla -Free subversion Hosting- http://www.assembla.com}
\bibitem{cita-thompson}
\newblock \url{http://boole.computer.org/portal/site/computer/menuitem.eb7d70008ce52e4b0ef1bd108bcd45f3/index.jsp?&pName=computer_level1&path=computer/homepage/0599/thompson&file=thompson.xml&xsl=article.xsl&},Unix and Beyond: An Interview with Ken Thompson ,visitado Domingo 23 de agosto de 2009
\end{thebibliography}

%----------------------------------------------------------------------

\end{document}
