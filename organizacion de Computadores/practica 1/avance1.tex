%======================================================================
%----------------------------------------------------------------------
%sergiobuj's ieee format template
%======================================================================
\documentclass[%
	%draft,
	%submission,
	%compressed,
	final,
	%
	%technote,
	%internal,
	%submitted,
	%inpress,
	%reprint,
	%
	%titlepage,
	notitlepage,
	%anonymous,
	narroweqnarray,
	inline,
	twoside,
         %invited,
	]{ieee}

\newcommand{\latexiie}{\LaTeX2{\Large$_\varepsilon$}}

%Español
	\usepackage[utf8]{inputenc}
	\usepackage[spanish]{babel}
%loñapsE

%\usepackage{ieeetsp}	% if you want the "trans. sig. pro." style
%\usepackage{ieeetc}	% if you want the "trans. comp." style
%\usepackage{ieeeimtc}	% if you want the IMTC conference style

% Use the `endfloat' package to move figures and tables to the end
% of the paper. Useful for `submission' mode.
%\usepackage {endfloat}

% Use the `times' package to use Helvetica and Times-Roman fonts
% instead of the standard Computer Modern fonts. Useful for the 
% IEEE Computer Society transactions.
%\usepackage{times}
% (Note: If you have the commercial package `mathtime,' (from 
% y&y (http://www.yandy.com), it is much better, but the `times' 
% package works too). So, if you have it...
%\usepackage {mathtime}

% for any plug-in code... insert it here. For example, the CDC style...
%\usepackage{ieeecdc}

\begin{document}

%----------------------------------------------------------------------
% Title Information, Abstract and Keywords
%----------------------------------------------------------------------
\title[Avance práctica 1]{%
       Avance práctica 1 \\  Organización de Computadores}

% format author this way for journal articles.
% MAKE SURE THERE ARE NO SPACES BEFORE A \member OR \authorinfo
% COMMAND (this also means `don't break the line before these
% commands).
\author[]{Sebastián Arcila Valenzuela (\textit{sarcilav@eafit.edu.co}),
\and{} Ruben Dario Bueno Angel (\textit{rbuenoan@eafit.edu.co})
\and{}y Sergio Botero Uribe (\textit{sbotero2@eafit.edu.co}).
}

% format author this way for conference proceedings
%\author[PLETT AND KOLL\'{A}R]{%
        %Gregory L. Plett\member{Student Member},\authorinfo{%
        %Department of Electrical Engineering,\\ 
        %Stanford University, Stanford, CA 94305-9510.\\
        %Phone: $+$1\,650\,723-4769, email: glp@simoon.stanford.edu}%
%\and{}and%
%\and{}Istv\'{a}n Koll\'{a}r\member{Fellow}\authorinfo{%
        %Department of Measurement and Instrument Engineering,\\ 
        %Technical University of Budapest, 1521 Budapest, Hungary.\\
        %Phone: $+$\,36\,1\,463-1774, fax: +\,36\,1\,463-4112, 
        %email: kollar@mmt.bme.hu}
%}

\journal{ST0254-031 Organización de Computadores}
\titletext{, \today}
%\ieeecopyright{0018--9456/97\$10.00 \copyright\ 1997 IEEE}
%\lognumber{xxxxxxx}
%\pubitemident{S 0018--9456(97)09426--6}
%\loginfo{Manuscript received September 27, 1997.}
%\firstpage{0}

%\confplacedate{Ottawa, Canada, May 19--21, 1997}

\maketitle               

\begin{abstract} 
Avance número 1 de la práctica 1, funciones trigonométricas en lenguaje ensamblador, trabajo realizado desde que se publicó el enunciado de la práctica hasta el 31 de julio de 2009.
\end{abstract}

\begin{keywords}
Práctica 1, organización de computadores, ensamblador, funciones trigonométricas, intel, fasm
\end{keywords}

%----------------------------------------------------------------------
% SECTION I: Introduction
%----------------------------------------------------------------------
\section{Comentarios}

\PARstart En este primer avance de la prácitca que se está programando  
para linux, se entrega el código en C de la aplicación
que usará la librería, aplicación con GTK para la interfaz gráfica.
Además un primer acercamiento al código de las funciones trigonométricas.
Se ha optado por realizar una serie de modificaciones para usar
solamente una función que cumpla la tarea de las funciones de seno
y coseno, además de unos cambios en la forma de calcular el factorial y
la exponenciación para eliminar la necesidad de otras funciones o ciclos adicionales.\\
Por ahora la aplicación en C no está conectada con la librería.

%----------------------------------------------------------------------
% SECTION II: Estructura de la práctica
%----------------------------------------------------------------------
\section{ Estructura de la práctica}

El código y documentos de este primer avance se encuentra en el archivo
comprimido con las siguientes carpetas y caracteristicas:
\begin{description}
\item[app:] En esta carpeta se encuentra el código en C de la
interfaz gráfica junto con el archivo xml usado para generar todos los
componentes de la ventana. También está el archivo Makefile para
compilar la aplicación.

\item[Doc:] Esta carpeta contiene toda la documentación digital usada
para realizar la práctica. (No se enviarán las referencias en este avance para
evitar que el archivo quede muy grande)
\item[ejemplos\_c:] Implementacion de las funciones como sumatorias
en C, además de código para probar las modificaciones pensadas para
simplificar el código final.
\item[lib:] En esta carpeta se guardará la librería que será cargada por la aplicación,
todavía no contiene el archivo final.\\
\end{description}



%----------------------------------------------------------------------
% SECTION III: Dificultades
%----------------------------------------------------------------------
\section{Dificultades}

La principal dificultad que se ha tenido y que no ha permitido que la práctica
avance un poco más rápido ha sido comprender la forma en que trabaja
el lenguaje ensamblador y entender que no se tiene compilador que haga 
algún trabajo extra.\\
Se han tenido dificultades con las
declaraciones que se deben dar para indicar tipos de datos, precisión y demás funciones
que debe conocer el procesador para ejecutar el código.
\\ 
%----------------------------------------------------------------------
\section{Tareas por realizar}

Lista de las actividades que faltan para terminar la práctica y
que además serían interesantes realizar:
\begin{itemize}
\item Entender a fondo las instrucciones de lenguaje
ensamblador utilizadas.
\item Ejemplos usando fasm para apropiarse de las
instrucciones usadas.
\item Terminar el código en lenguaje ensamblador.
\item Hacer optimizaciones y pulir el código.
\item Hacer pruebas compilando el programa con 
la librería funcionando propiamente.
\end{itemize}

%----------------------------------------------------


\begin{thebibliography}{1}

\bibitem{Assembla -Free subversion Hosting}
\newblock {\em http://www.assembla.com/spaces/dashboard/index/trigo\_assembla},
\newblock Sitio dondese encuentra todo el desarrollo y actividades de la práctica.
\newblock{Assembla -Free subversion Hosting- http://www.assembla.com}

\end{thebibliography}

%----------------------------------------------------------------------

\end{document}
